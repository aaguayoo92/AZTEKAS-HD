%%%%%%%%%%%%%%%%%%%%%%%%%%%%%%%%%%%%%%%%%%%%%%%%%%%%%%%%%%%%%%%
%%%%%%%%%%%%%%%%%%%%%%%%%% Hydrodynamics %%%%%%%%%%%%%%%%%%%%%%
%%%%%%%%%%%%%%%%%%%%%%%%%%%%%%%%%%%%%%%%%%%%%%%%%%%%%%%%%%%%%%%
\section{Hydrodynamics}
\label{sec:HD}

The Euler equations in the non-relativistic regime written in a balanced
form are~\citep[see][]{landau1987} (using Einstein notation):

%%%%%%%%%%%%%%%
%% Equations %%
%%%%%%%%%%%%%%%
\begin{equation}
    \frac{\partial \rho}{\partial t} + \nabla_j \left( \rho u^j \right) = 0,
\end{equation}
\begin{equation}
    \frac{\partial E}{\partial t} + \nabla_j \left[ (E + p) u^j \right] = \Gamma
    - \Lambda + f_j u^j,
\end{equation} 
\begin{equation}
    \frac{\partial \left( \rho u^i \right)}{\partial t} + \nabla_j \left( \rho
    u^i u^j \right) + \nabla^i p = f^i,
\end{equation} 

\noindent where $\rho$ is the mass density, $u^i = \mathrm{d}x^i/\mathrm{d}t$ is
the velocity vector, $p$ is the pressure and $E$ is the total energy density.
We define the covariant derivative $\nabla_i$
inverse three-dimensional metric tensor is defined as
$\gamma^{ij}$, and so $\nabla^j = \gamma^{ij} \nabla_i$, $\Gamma -
\Lambda$ represents the net energy gain/loss per unit volume and $\vec{f}$
are the external forces per unit of mass acting on the fluid, such as gravity.

In order to close the system, we need an equation for the total energy
density. In general,

%%%%%%%%%%%%%%
%% Equation %%
%%%%%%%%%%%%%%
\begin{equation}
    E = \frac{1}{2}\rho\, (u_j u^j) + \rho \epsilon,
\end{equation}

\noindent where we define $u_i = \gamma_{ij}u^j$ and $\epsilon$ is the internal energy of
the fluid, which follows an equation of state (EoS) of the form $\epsilon =
\epsilon(\rho,p)$. For \textit{aztekas}, we adopted a polytropic EoS:
%%%%%%%%%%%%%%
%% Equation %%
%%%%%%%%%%%%%%
\begin{equation}
    \epsilon = \frac{p}{\rho (\gamma - 1)},
\end{equation}

\noindent where $\gamma$ is the polytropic index of the gas. Different and
more realistic both analytic and numerical EOS had been developed to study
either gases~\citep[cf.][]{ryu2006} or more complex systems like neutron
stars~\citep[cf.][]{zhu2018}, but the polytropic gas approximation remains
a valid and good approximation for many astrophysical phenomena.

The primitive and conservative variables, as well as the fluxes and sources
terms for this system, using index  $i$ to represent one specific flux
direction and $j$ all three spatial directions, are:

%%%%%%%%%%%%%%%
%% Equations %%
%%%%%%%%%%%%%%%
\begin{equation}
    \mathbf{u} = \left( \rho, p, v^j \right),
\end{equation}
\begin{equation}
    \mathbf{q} = \left( \rho, E, \rho v^j \right),
\end{equation}
\begin{equation}
    \mathbf{f}^i = \left( \rho v^i, [E + p]v^i, \rho v^i v^j + p\delta^{ij} \right),
\end{equation}

\noindent and

\begin{equation}
    \mathbf{s} = \left( 0, \Gamma - \Lambda + \vec{f}\cdot \vec{v}, f^j \right).
\end{equation}

\noindent where $v^j$ is the physical velocity vector, which is related with $u^j$ as
\begin{equation}
    v^j = \sqrt{\gamma_{(jj)}} u^j,
\end{equation}

\noindent where $\gamma_{(jj)}$ are the three-dimensional metric components along the diagonal (no implicit sumation is intended).
