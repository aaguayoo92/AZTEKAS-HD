%%%%%%%%%%%%%%%%%%%%%%%%%%%%%%%%%%%%%%%%%%%%%%%%%%%%%%%%%%%%%
%%%%%%%%%%%%%%%%%%%%%%%% Introduction %%%%%%%%%%%%%%%%%%%%%%%
%%%%%%%%%%%%%%%%%%%%%%%%%%%%%%%%%%%%%%%%%%%%%%%%%%%%%%%%%%%%%

\section{Introduction}
\label{sec:intro}


The study of fluid dynamics represents an important
part for the understanding of our universe. Although gravity is the
dominant force in the universe, it is the gas and the plasma, and their
interaction with the gravitational and magnetic fields, the responsible
of a very wide range of astrophysical phenomena. The plasma in the
universe  is the most abundant state of matter.   It can 
be found from stellar up to cosmological scales and the gas dynamics
allows a comprehensive study for  its evolution and behavior.  The
understanding of the coupling with magnetic
fields and radiative processes due to the micro physics of its particles is
essential to model different astronomical observations.

The equations that describe the dynamics of flows are described by the
Euler equations. They constitute a set of non-linear system of hyperbolic
partial differential equations that in most cases have no exact solutions.
This is something very common in physical problems dealing with a set
of partial differential equations.  Usually exact analytic solutions
can be found in cases of highly idealized, symmetrical or self-similar
problems~\citep[see for example][]{landau1987,sedov1959,toro2009,edgar2004} %(\textcolor{red}{CITA
%A LANDAU, ZELDOVICH, TORO, SEDOV, EDGAR}).  
However, in the general
case where more complex scenarios appear, these equations have to be
solved numerically.

Numerical hydrodynamics and, in general, the study of hyperbolic
partial differential equations, constitutes a full branch of study
in physics. Some relevant introductory literature in this topic are the
books by \citet{toro2009}, \citet{leveque2002} and \citet{laney1998},
\textcolor{red}{MAS}.  \textcolor{blue}{The appendix of the article by \citet{aguayo2018} is
a simple comprehensive way in which finite volume methods applied to fluid
dynamics in astrophysics is described.} The continuous search for new and better algorithms to
solve these equations has lead to the development of robust and more accurate
computational codes. These have been used to test and further extend
theoretical models that continuously enrich our understanding of the universe.

In the special case of hydrodynamics, one of the first numerical codes
to be extensively used for astrophysical hydrodynamics is the ZEUS
code~\citep{stone1992}, which uses a finite difference scheme for the
discretization of the differential equations.  Even more, ZEUS has a GNU
Public License (GPL) for its distribution which through the years has made it a
first start for many researchers.  Nowadays more accurate and efficient
numerical methods have been developed reducing computational costs.  Some
of the current hydrodynamical codes that are available to the community are
the following:

\begin{itemize}
  \item PLUTO, reference, description and license.
  \item FLASH, the license use is very restrictive
\end{itemize}

  At the very end it is not only important to have a very robust
hydrodynamical code, but also make it freely available (with a GNU GPL
license or equivalent) to the community, ideally expecting more programmers
to join forces to further improve it.

  In this article we are not aiming at presenting a new different
state-of-the-art code.  Our intention is 
to present and validate our free software (open source, open access)
\textit{aztekas} which has a GNU GPL license, with the aim 
to invite the community to use it and contribue to its development. We also 
include in this article several standard benchmark tests for code
validation, together with the precise form of the algorithms used for its
construction. 


